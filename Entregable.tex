\documentclass[12pt, a4paper]{article}

% Text languages
\usepackage[spanish, english, UKenglish, USenglish, american, british]{babel}

% Accents
\usepackage[latin1]{inputenc}

% Maths
\usepackage{mathtools}
\usepackage{amsmath,amsthm,amssymb}

% Math-mode symbol & verbatim
\def\W#1#2{$#1{#2}$ &\tt\string#1\string{#2\string}}
\def\X#1{$#1$ &\tt\string#1}
\def\Y#1{$\big#1$ &\tt\string#1}
\def\Z#1{\tt\string#1}

% A non-floating table environment.
\makeatletter
\renewenvironment{table}%
   {\vskip\intextsep\parskip\z@
    \vbox\bgroup\centering\def\@captype{table}}%
   {\egroup\vskip\intextsep}
\makeatother

\DeclarePairedDelimiter\abs{\lvert}{\rvert}%
\DeclarePairedDelimiter\norm{\lVert}{\rVert}%

% Swap the definition of \abs* and \norm*, so that \abs
% and \norm resizes the size of the brackets, and the 
% starred version does not.
\makeatletter
\let\oldabs\abs
\def\abs{\@ifstar{\oldabs}{\oldabs*}}
%
\let\oldnorm\norm
\def\norm{\@ifstar{\oldnorm}{\oldnorm*}}
\makeatother


% https://www.overleaf.com/learn/latex/Page_size_and_margins
\usepackage{geometry}
\topmargin = -23pt
\oddsidemargin = 13pt
\headheight = 12pt
\headsep = 25pt
\textheight = 674pt
\textwidth = 426pt
\marginparsep = 10pt
\marginparwidth = 50pt
\footskip = 30pt
\marginparpush = 5pt
\hoffset = 0pt
\voffset = 0pt
\paperwidth = 597pt
\paperheight = 845pt

% Hyperlinks
\usepackage{hyperref}

% Figure
\usepackage{graphicx}
% \usepackage{subcaption}
\usepackage{etoc}
% Example
\newtheorem{exmp}{Example}[section]
% Algorithms
%\usepackage[]{algorithm2e}
%\usepackage{algorithm}% http://ctan.org/pkg/algorithm
%\usepackage{algpseudocode}% http://ctan.org/pkg/algorithmicx
\usepackage{algpseudocode}

\renewcommand{\thefootnote}{\arabic{footnote}} % 1, 2, 3... (la que hay por defecto)

\usepackage{titlesec}
\setcounter{secnumdepth}{5}

\titleformat{\paragraph}
{\normalfont\normalsize\bfseries}{\theparagraph}{1em}{}
\titlespacing*{\paragraph}
{0pt}{3.25ex plus 1ex minus .2ex}{1.5ex plus .2ex}

\usepackage{float}
%--------------------------------------------------------------------------
\title{Detection of/between similarity of documents with hashing}
\author{Roger Vilaseca Darné, Xavier Lacasa Curto and Xavier Martín Ballesteros\\
  \small Algorithms\\
}
\date{10th December 2018}

\begin{document}
% Images
\graphicspath{ {./images/}, {./plots/} }

%\maketitle

\begin{titlepage}
	\centering
	%{\scshape\LARGE UNIVERSITAT POLITÈCNICA DE CATALUNYA \par}
	\vspace{1cm}
	{\scshape\Large ALGORTIHMS\par}
	\vspace{1.5cm}
	{\huge\bfseries Detection of similarity between documents with hashing\par}
	\vspace{2cm}
	{\Large\itshape Roger Vilaseca Darné, Xavier Lacasa Curto and Xavier Martín Ballesteros\par}
	\vfill
%	\includegraphics[width=0.15\textwidth]{UPC.png}\par\vspace{1cm}
	%supervised by\par
	%Dr.~Mark \textsc{Brown}

	\vfill

% Bottom of the page
	{\large 10th December 2018}
\end{titlepage}

%\abstract{Esto es una plantilla simple para un articulo en \LaTeX.}

%	*********************** ÍNDEX *********************
\setcounter{secnumdepth}{5}
\setcounter{tocdepth}{5}

\newpage
  \tableofcontents
\newpage

\section{Introduction}

% Referència a una equació \ref{eq:area}).
% Referència a una secció \ref{sec:nada}
% Referència a una cita \cite{Cd94}.

\begin{center}
 \begin{tabular}{|c || c | c | c|} 
 \hline
   & boada-1 \textbf{to} boada-4 & boada-5 & boada-6 \textbf{to} boada-8 \\
 \hline
 Number of sockets per node & 2 & 2 & 2 \\ 
 \hline
 Number of cores per socket & 6 & 6 & 8 \\
 \hline
 Number of threads per core & 2 & 2 & 1 \\
 \hline
 Maximum core frequency & 2395 MHz & 2600 MHz & 1700 MHz \\
 \hline \hline
 L1-I cache size (per-core) & 32 kB & 32 kB & 32 kB \\ 
 \hline
 L1-D cache size (per-core) & 32 kB & 32 kB & 32 kB \\
 \hline
 L2 cache size (per-core) & 256 kB & 256 kB & 256 kB \\
 \hline
 Last-level cache size (per-socket) & 12288 kB & 15360 kB & 20480 kB \\
 \hline \hline
 Main memory size (per socket) & 12 GB & 31 GB & 16 GB \\
 \hline
 Main memory size (per node) & 23 GB & 63 GB & 31 GB \\
 \hline
\end{tabular}
\end{center}

\begin{center}
 \begin{tabular}{|c || c | c | c|} 
 \hline
 \textbf{Version} & T$_1$ & T$_\infty$ & \textbf{Parallelism} \\ 
 \hline\hline
 seq & 6 & 87837 & 787 \\ 
 \hline
 v1 & 7 & 78 & 5415 \\
 \hline
 v2 & 545 & 778 & 7507 \\
 \hline
 v3 & 545 & 18744 & 7560 \\
 \hline
 v4 & 88 & 788 & 6344 \\ 
 \hline
 v5 & as & as & adf \\
 \hline
\end{tabular}
\end{center}

\begin{center}
 \begin{tabular}{|c || c | c || c | c | c|} 
 \hline
 \textbf{Version} & $\phi$ & S$_\infty$ & T$_1$ & T$_8$ & S$_8$ \\
 \hline\hline
 initial version in \textbf{3dfft\_omp.c} & 6 & 87837 & 787 & 6 & 87837 \\ 
 \hline
 new version with improved $\phi$ & 7 & 78 & 5415 & 6 & 87837 \\
 \hline
 final version with reduced parallelisation overheads & 545 & 778 & 7507 & 6 & 87837 \\
 \hline
\end{tabular}
\end{center}

% Bibliografía.
%-----------------------------------------------------------------
\newpage
\begin{thebibliography}{9}
\bibitem{The Best} 
Anand Rajaraman, Jure Leskovec and Jeffrey D. Ullman. 
\textit{Mining of Massive Datasets}. Cambridge University Press. (December 30, 2011).
 
 \bibitem{Bottom} 
 cmhteixeira. \textit{Locality Sensitive Hashing (LSH)} [online]. (November 29, 2017). $<$https://aerodatablog.wordpress.com/2017/11/29/locality-sensitive-hashing-lsh/$>$[Consulted: December 12, 2018].
 
 \bibitem{Wtf is this name}
 Hubert Brylkowski. \textit{Locality sensitive hashing $-$LSH explained} [online]. (October 6, 2017). $<$https://medium.com/engineering-brainly/locality-sensitive-hashing-explained-304eb39291e4$>$[Consulted: December 15, 2018].
 
 \bibitem{Ma maaan}
 Jeffrey D. Ullman[Mining Massive Datasets]. (July 23, 2016). \textit{3 2 Minhashing 25 18}. $<$https://www.youtube.com/watch?v=96WOGPUgMfw$>$.
 
 \bibitem{Ma man agaaaain}
 Jeffrey D. Ullman[Mining Massive Datasets]. (July 23, 2016). \textit{3 3 Locality Sensitive Hashing 19 24}. $<$https://www.youtube.com/watch?v=\_1D35bN95Go$>$.
 
 \bibitem{Second one}
 santhoshhari. \textit{Locality Sensitive Hashing: Application of Locality Sensitive Hashing to Audio Fingerprinting} [online]. (n. d.). $<$https://santhoshhari.github.io/Locality-Sensitive-Hashing/$>$[Consulted: December 16, 2018].
 
 \bibitem{The firsty one}
  Shikhar Gupta. \textit{Locality Sensitive Hashing: An effective way of reducing the dimensionality of your data} [online]. (June 29, 2018). $<$https://towardsdatascience.com/understanding-locality-sensitive-hashing-49f6d1f6134$>$[Consulted: December 16, 2018]
 
%\url{http://www.mit.edu/~andoni/LSH/} 
 
\end{thebibliography}

\end{document}
